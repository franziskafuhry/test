\documentclass[
	pdftex,
	fontsize=12pt,
	paper=a4,
	parskip=half,
	twoside=false,
	numbers=noenddot,	% Kein Punkt am Ende einer Überschrift
	%draft=true,			% Deckt Schwächen auf: overfull und full boxes werden markiert; Bilder werden nicht geladen
	bibliography=totoc,	% Literaturverzeichnis ins Inhaltsverzeichnis aufnehmen
	listof=totoc,		% Tabellen- und Abbildungsverzeichnis ins Inhaltsverzeichnis aufnehmen
	titlepage=true,		% Separate Titelseite; Gestaltung mit Hilfe der Titlepage-Umgebung
	headsepline=true,	% Kopflinie aktivieren
	footsepline=true,	% Fußlinie aktivieren
	abstracton			% Abstract aktivieren
]{scrreprt}

% Abstand zwischen Überschrift und Text
%evtl... 

% Absatzgröße:
\setlength{\parskip}{6pt}

% Zeichenkodierung Input ist UTF-8: Umlaute können direkt eingegeben werden
\usepackage[utf8]{inputenc}

% nötig für tabelle (Im Titelblatt!)
\usepackage{longtable}

% variabel breite tabellen
\usepackage{tabularx}

% Fußnoten in Tab
\usepackage{tablefootnote}

\usepackage{caption}

% Zeichenkodierung Ausgabe ist T1-Kodierung: Wichtig für die Ausgabe von Umlauten
\usepackage[T1]{fontenc}

% Schrift festlegen
\usepackage{lmodern}

% Sprachauswahl für Lokalisierungen und Silbentrennung
\usepackage[ngerman, english]{babel}

% Zitate: Anführungszeichen automatisch anhand der Sprache wählen
\usepackage[babel=true]{csquotes}

% Source-Code-Listings
\usepackage{listings}
\lstset{numberbychapter=false}
%\lstset{language=SQL}

% BibTeX-Symbol etc.
\usepackage{texnames}

%Euro-Symbol
\usepackage{eurosym}

%Mathematische Symbole
\usepackage{amssymb}

% (Schrift-) Farben ändern
\usepackage{xcolor}

% Symbole, z.B. Haken
%\usepackage{pifont}

% Mathe-definitionen
%\usepackage{amsthm}

% Zeilen in Tabellen zusammenfassen
%\usepackage{multirow}

% Silbentrennung kann bei bestimmten Wörten mit Hilfe von diesem Paket deaktiviert werden 
%\usepackage{hyphenat}

% Abkürzungsverzeichnis
%option printonly used + withpage gibt auch die Seitenzahlen an
\usepackage[printonlyused]{acronym}
% Abstand mit Punkten füllen - VERALTET NICHT MEHR UNTERSÜTZT!
%\newcommand*\bflabel[1]{\textbf{\normalsize{#1}}\hfill}

% Punkte im Inhaltsverzeichnis
\usepackage{tocstyle}
\usetocstyle{allwithdot}

% Zum Einbinden von PDF-Dateien.
%\usepackage{pdfpages}

% Paket zum Anpassen von Kopf- und Fußzeilen
\usepackage[plainfootsepline, plainheadsepline, headsepline, footsepline, automark]{scrpage2}

% Liniendicke
\setheadsepline{0.1pt}
\setfootsepline{0.1pt}

% Kopf- und Fusszeile löschen
\clearscrheadfoot
% Kopf- und Fusszeile aktivieren
\pagestyle{scrheadings}

% Kopf links
\ihead[\titel]{\titel}

% Fuss links
\ifoot[\verfasser]{\verfasser}
% Fuss rechts
\ofoot[\pagemark]{\pagemark}

% Grafiken einbinden
\usepackage{graphicx}
% Pfad zu den Grafiken
\graphicspath{{Figures/}}

% Seitenränder setzen
\usepackage[left=3.5cm, right=2.5cm, top=2.5cm, bottom=2cm, footskip=1.25cm]{geometry}

% Zeilenabstand auf 1.5 setzen
\usepackage{setspace}
\onehalfspacing

% Literaturverzeichnis
\usepackage[backend=biber, style=alphabetic]{biblatex}
%für Umbrüche: block=ragged
\addbibresource{Bibliography/Bibliography.bib}

% Glossar
%\usepackage[toc,nonumberlist]{glossaries}

% Titel als Referenzierung verwenden
\usepackage{titleref}

% Währungen
%\usepackage{textcomp}

% Fussnoten und Bilder fortlaufend nummerieren.
\usepackage{chngcntr}
\counterwithout{table}{chapter}
\counterwithout{figure}{chapter}

% Persönliche Daten
\newcommand{\titel}{Mein Titel}
%\newcommand{\untertitel}{}
\newcommand{\art}{Art der Arbeit}
\newcommand{\verfasser}{Verfasser}
\newcommand{\kurs}{Kurs oder sowas}
\newcommand{\ausbildungsbetrieb}{rauchst du wahrscheinlich nichtb}%, Dietmar-Hopp-Allee 16, 69190 Walldorf}
\newcommand{\abgabedatum}{Datum}

\newcommand{\martrikelnr}{Matrikelnummer}
\newcommand{\wissenschaftlBetreuer}{asdf}
\newcommand{\betrieblBetreuer}{asdf}
\newcommand{\zeitraum}{13.02.2017 -- 22.05.2017}

% Links- und PDF-Einstellungen
\usepackage[breaklinks, bookmarksopen=true, pdfpagelabels]{hyperref}
\hypersetup{
	pdfauthor = {\verfasser},
	pdftitle = {\titel},
	pdfsubject = {\art},
	pdfkeywords = {},
	pdfstartview = {Fit},
	colorlinks = {false}
}

% Verhinderung von Schusterjunge und Hurenkind
\clubpenalty = 10000
\widowpenalty = 10000
\displaywidowpenalty = 10000

% Seitenzäler für große, römische Zahlen
\newcounter{RomanPagenumber}

% Abkürzungen
%\newcommand{\dash}{d.\,h.}
%\newcommand{\zB}{z.\,B.}

%Beispielseite mit Rändern und Größen
%kann mit "\layout" in das document eingefügt werden
\usepackage{layout}

%Zeigt Boxen an:
%\usepackage{showframe}

%Für PSTricks
%\usepackage{pstricks}
